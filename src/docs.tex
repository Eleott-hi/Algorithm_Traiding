\documentclass[12pt]{article}
\usepackage{amsmath}
\usepackage{graphicx}
\usepackage{hyperref}
\usepackage[latin1]{inputenc}

\title{Algorithmic Trading}
\author{cgreenbe, pintoved}
\date{2/1/2023}

\begin{document}
\maketitle

\newpage
\section{brief introduction}
Plot the probabilistic behavior of stock quotes and their price forecasts for the foreseeable future 
based on historical data without regard to external factor
using interpolation and approximation algorithms.

Part 1. Interpolation of tabulated functions
\begin{itemize}
  \item Drawing the cubic spline graph;
  \item Drawing the graph by the Newton polynomial of nth degree;
  \item There can be up to 5 graphs displayed in the field at the same time.
\end{itemize}

Part 2. Approximation of tabulated functions
\begin{itemize}
  \item Plot a tabulated function of stock quotes using the least squares method;
  \item The user sets the number of days for which we want to extend the graph;
  \item Drawing the graph plotted by the polynomial of the degree set at that time;
  \item There can be up to 5 graphs with the same value of the number of days displayed at the same time.
\end{itemize}

Part 3. Bonus. Research on temporal characteristics
\begin{itemize}
  \item Measure the time required to calculate values using the spline interpolation algorithms
  and Newton's method, depending on the change in the number of points.
\end{itemize}

Part 4. Bonus. Approximation with weights
\begin{itemize}
  \item The method of least squares takes into account the weights of points.
\end{itemize}

\end{document}
